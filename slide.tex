\documentclass{anu-slide} % Adapted for ANU branding

% For filler text:
\usepackage[base]{babel}
\usepackage{lipsum}
\usepackage{graphicx}


\renewcommand{\titlelogo}{\includegraphics[width=.3\paperwidth,keepaspectratio]{logo/anu.png}}
\renewcommand{\slidelogo}{\includegraphics[width=.05\paperwidth,keepaspectratio]{logo/anu_ crest_gold.png}}
% For non-CS ANU logo use the lines below:
% \renewcommand{\titlelogo}{\includegraphics[width=.3\paperwidth,keepaspectratio]{logo/anu-rev.png}}
% \renewcommand{\slidelogo}{\includegraphics[width=.3\paperwidth,keepaspectratio]{logo/anu.png}}
% Ensure correct logo application per ANU guidelines

\title{An ANU \LaTeX\ Slide Template}
\subtitle{an has this subtitle}
\author{An ANU person\texorpdfstring{\footnotemark[1]}{}, A Secondary Author\texorpdfstring{\footnotemark[2]}{}}
\institute{\texorpdfstring{\footnotemark[1]}{}Australian National University, \texorpdfstring{\footnotemark[2]}{}Collaborating Institution}
\date{Last Compiled: \today}
\renewcommand{\slidefoot}{\LaTeX\ footnotes are pretty}

\begin{document}

\begin{titleframe}{}
    \maketitle
\end{titleframe}

\begin{titleframe}{Text in \LaTeX}
    Examples of Basic Text Typesetting
\end{titleframe}

\section{Long Text}

\begin{frame}{This is a Really Long Text of Title Used Here}
    Some really long text:
    
    \bigskip
    
    \lipsum[2]
\end{frame}

\section{List}

\begin{frame}{Itemize List}
    Some introduction of the list.
    \begin{itemize}
        \item Bulleted copy. Keep it short with bite-size chunks of information.
        \begin{itemize}
            \item Bulleted copy. Keep it short with bite-size chunks of information.
        \end{itemize}
        \item Bulleted copy. Keep it short with bite-size chunks of information.
        \pause\item Bulleted copy on the second slide. Keep it short with bite-size chunks of information.
    \end{itemize}
\end{frame}

\begin{frame}{Enumerate List}
    Some introduction of the list.
    \begin{enumerate}
        \item Bulleted copy. Keep it short with bite-size chunks of information.
        \begin{enumerate}
            \item Bulleted copy. Keep it short with bite-size chunks of information.
        \end{enumerate}
        \item Bulleted copy. Keep it short with bite-size chunks of information.
        \item Bulleted copy. Keep it short with bite-size chunks of information.
    \end{enumerate}
\end{frame}

\begin{titleframe}{Features in \LaTeX}
    Examples of Features Commonly Used in Slides
\end{titleframe}

\section{Maths}

\begin{frame}{Maths}
    An example of some very long equations with $\Psi (x,t)$:
    
    \begin{align}
    i\hbar {\frac {\partial }{\partial t}}\Psi (x,t)&=\left[-{\frac {\hbar ^{2}}{2m}}{\frac {\partial ^{2}}{\partial x^{2}}}+V(x,t)\right]\Psi (x,t) \\
    i\hbar {\frac {d}{dt}}\vert \Psi (t)\rangle &={\hat {H}}\vert \Psi (t)\rangle \\ 
    |\Psi (t)\rangle &=\sum _{n}A_{n}e^{{-iE_{n}t}/\hbar }|\psi _{E_{n}}\rangle
    \end{align}
    
    Indeed an example of some very long equations with $\Psi (x,t)$.
\end{frame}

\section{Figure}
\begin{frame}{Figure}
    \LaTeX\ can draw figures with the tikz package:
    \begin{figure}[h]
    \centering
    \begin{tikzpicture}[
    square/.style={rectangle, draw=black!, very thick, minimum size=5mm},
    ]
    \node[square](client){\footnotesize Client};
    \node[square](r1)[right=of client]{\footnotesize Relay 1};
    \node[square](r2)[right=of r1]{\footnotesize Relay 2};
    \node[square](r3)[right=of r2]{\footnotesize Relay 3};
    \node[square](server)[right=of r3]{\footnotesize Server};
    \draw[->](client.east)--node[above=1em]{\footnotesize$E_1(E_2(E_3(P)))$}(r1.west);
    \draw[->](r1.east)--node[above=1em]{\footnotesize$E_2(E_3(P))$}(r2.west);
    \draw[->](r2.east)--node[above=1em]{\footnotesize$E_3(P)$}(r3.west);
    \draw[->](r3.east)--node[above=1em]{\footnotesize$P$}(server);
    \end{tikzpicture}
    \caption{An Example of a Three-Hop Connection}
    \label{fig:three-hop}
    \end{figure}
\end{frame}

% Remaining frames and content unchanged

\end{document}
